\section{Results}
The inferred redshift corrections at the S2 pericenter from its orbit parameters and the predicted corrections of the RAR model are compared in \cref{fig:redshift}. Note that the considered RAR solutions consider a slightly larger central core mass ($M_c = \SI{4.2E6}{\Msun}$) compared to $M_c \approx \SI{4.1E6}{\Msun}$ as obtained by \citet{2018A&A...615L..15G}. Nevertheless, the results show that mass distributions of the RAR model are able to reproduce the relativistic redshift correction. However, note that for the solution corresponding to the minimal DM particle mass ($mc^2 = \SI{48}{\kilo\eV}$) the redshift prediction shows a little discrepancy with respect to the reference value due to the close proximity of the S2 pericenter to the \textit{semi-surface}\footnote{Is ok to call it a surface?} of the quantum core.

\loadfigure{figures/redshift}